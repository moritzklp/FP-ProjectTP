\documentclass[12pt,a4paper]{article}
\usepackage{etex,datetime,setspace,latexsym,amssymb,amsmath,amsthm}
\usepackage{fancybox,dialogue,float,wrapfig,enumerate,microtype}
\usepackage{verbatim,xcolor,multicol,titlesec,tabularx,mdframed}

\usepackage[utf8]{inputenc}
\usepackage[pdftex]{hyperref}
\usepackage[margin=2cm,bottom=3cm,footskip=15mm]{geometry}
\parindent0cm
\parskip0.2em

\usepackage{tikz}
\usetikzlibrary{arrows,trees,positioning,shapes,patterns}
\usetikzlibrary{intersections,calc,fpu,decorations.pathreplacing}

\usepackage[T1]{fontenc} % better fonts

% Haskell code listings in our own style
\usepackage{listings,color}
\definecolor{lightgrey}{gray}{0.35}
\definecolor{darkgrey}{gray}{0.20}
\definecolor{lightestyellow}{rgb}{1,1,0.92}
\definecolor{dkgreen}{rgb}{0,.2,0}
\definecolor{dkblue}{rgb}{0,0,.2}
\definecolor{dkyellow}{cmyk}{0,0,.7,.5}
\definecolor{lightgrey}{gray}{0.4}
\definecolor{gray}{gray}{0.50}
\lstset{
  language        = Haskell,
  basicstyle      = \scriptsize\ttfamily,
  keywordstyle    = \color{dkblue},     stringstyle     = \color{red},
  identifierstyle = \color{dkgreen},    commentstyle    = \color{gray},
  showspaces      = false,              showstringspaces= false,
  rulecolor       = \color{gray},       showtabs        = false,
  tabsize         = 8,                  breaklines      = true,
  xleftmargin     = 8pt,                xrightmargin    = 8pt,
  frame           = single,             stepnumber      = 1,
  aboveskip       = 2pt plus 1pt,
  belowskip       = 8pt plus 3pt
}
\lstnewenvironment{code}[0]{}{}

% only shown, not compiled:
\lstnewenvironment{showCode}[0]{\lstset{numbers=none}}{}

% only compiled, not shown:
\newcommand{\hide}[1]{}

% will the real phi please stand up
\renewcommand{\phi}{\varphi}

% load hyperref as late as possible for compatibility
\usepackage[pdftex]{hyperref}
\hypersetup{
  pdfborder = {0 0 0},
  breaklinks = true,
  linktoc = all,
}
\pdfinfoomitdate=1
\pdftrailerid{}
\pdfsuppressptexinfo15


\title{Classical Cryptography Report}
\author{Eva Imbens, Chiara Michelutti, Jan Przystał, Moritz Klopstock}
\date{\today}
\hypersetup{pdfauthor={Me}, pdftitle={My Report}}

\begin{document}

\maketitle

\begin{abstract}
We give a toy example of a report in \emph{literate programming} style.
The main advantage of this is that source code and documentation can
be written and presented next to each other.
We use the listings package to typeset Haskell source code nicely.
\end{abstract}

\vfill

\tableofcontents

\clearpage

% We include one file for each section. The ones containing code should
% be called something.lhs and also mentioned in the .cabal file.

\section{Introduction}
Cryptography has long been a cornerstone of secure communication, and among its many techniques, the One-Time Pad (OTP) cipher holds a special place. When implemented correctly, the OTP offers perfect secrecy by using a truly random key that is as long as the message itself. However, this ideal is contingent on using each key only once. When a key is reused—a situation known as the Many-Time Pad—the cipher’s security can be compromised, revealing vulnerabilities that attackers may exploit.

In this project, we look into the classical principles of cryptography by implementing the OTP cipher and investigating its limitations. Our approach not only involves the encryption and decryption of natural language messages but also includes the generation of secure random keys and a demonstration of the Many-Time Pad attack. By exposing the weaknesses that arise from improper key management, we aim to provide a comprehensive understanding of the balance between theoretical security and practical implementation challenges.

Haskell has been chosen as the programming language for this project due to the course dependencies but also its pure functional nature, strong static type system, and emphasis on code safety. These features are particularly advantageous in the realm of cryptographic operations, where the avoidance of unintended side effects is critical. Through this implementation, we explore the viability and effectiveness of Haskell in developing secure cryptographic solutions, leveraging its capabilities to create a reliable and efficient cipher system.

The following sections outline our project’s objectives, methodologies, and the experimental setup used to analyze the performance and security of our Haskell-based OTP implementation.


\subsection{Haskell Background}
\label{sec:why_haskell}
Haskell offers several advantages that make it a strong candidate for implementing cryptographic attacks, 
such as the Many-Time Pad attack. These benefits include performance, memory safety, and a strong type system. 
They all contribute to writing secure and reliable (cryptographic code). 

\paragraph{Compiled and Optimized Execution} 
Despite being a high-level functional language, Haskell can perform nearly as well as C. 
Research has shown that cryptographic functions implemented in Haskell can perform within the same order of magnitude as C, 
particularly when using compiler optimizations \cite{tevis2006secure}. 
This proves that it can handle computationally intensive cryptographic tasks, with the correct optimizations.

\paragraph{Lazy Evaluation} 
Haskell uses lazy evaluation, which means that it only computes the values when they are needed. 
This can help to improve the efficiency by avoiding unnecessary calculations. 
For our many-time pad attack this would help to handle large ciphertexts efficiently by processing them only when required. 
This also helps to reduce the memory usage and computation overhead.
Lazy Evaluation may however be a security concern. It can lead to timing attacks which may leak sensitive information \cite{lazy2013}.

\paragraph{Memory Safety}
Unlike languages like C, Haskell automatically manages the memory, preventing vulnerabilities such as buffer overflows and pointer-related bugs. 
This automatic memory management ensures that cryptographic operations do not suffer from unintended memory corruption. 
This is important in cryptographic applications because small memory errors can lead to security flaws. 

\paragraph{Strong Type System}
Haskell has a strong type system that ensures that variables hold only correct kinds of values. 
This prevents unintended operations, such as treating a byte array as a stringmisinterpreting cryptographic data formats. 
Programming on a type-level allows to encode security properties at compile time, which ensures that many classes of bugs are detected early.

\paragraph{Immutability}
Haskell's immutability ensures that once a value is assigned, it cannot be altered. 
This prevents unintended modifications of cryptographic data during the execution, 
which can be a problem in other languages where variables can be overwritten accidentally. 
Since cryptographic attacks and defenses often rely on maintaining strict data integrity, 
immutability provides a significant security advantage.

\paragraph{Arbitrary Precision Arithmetic}
Haskell provides arbitrary precision integers, 
which means that it allows computations with arbitrary large numbers which prevents the overflow issues that are common in many other languages. 
This is useful in cryptographic application where calculations may involve large integers, and unexpected overflows could lead to incorrect results.

\paragraph{Purity}
Haskell's pure functions make sure that the same input always produces the same output, 
which makes computations easier to test and debug. The lack of hidden side effects simplifies formal reasoning about cryptographic operations, 
which is useful in security audits and verification processes.



\input{exec/Pad.lhs}

\input{exec/Main.lhs}

\input{exec/MTP.lhs}
%\input{test/simpletests.lhs}

%
\section{Conclusion}\label{sec:Conclusion}

We have managed to implement encryption, decryption, for the Caesar, Vigenere, and One-Time Pad ciphers in Haskell. Additionaly our program allows for random key generation for all of them. We were also able to perform automated attacks, which allowed us to correctly guess the content of ciphertexts, for all implemented ciphers using brute force, frequency analysis, Kasiski examination and Friedmann tests, and Many-Time Pad. The program also contains a user interface to perform all those actions.

We have found Haskell's functional paradigm to match very well with how cphers work. They are based on cryptograhic functions which simply take an input and provide an output. A few operations are sometimes chained together, but the state never needs to be preserved which fits perfectly with Haskell.

\subsection{Future Work}
Our program could be improved to fully guess the words from partially discovered plaintexts, especially when performing the Many-Time Pad attack.

It would also be interesting to compare the performance of the Haskell implementations of the ciphers to other programming languages like C or Java. Cryptographic operations often need to be performed quickly in resource constrained systems so it would be interesting to see how well Haskell performs in such scenarios.

Finally, we can see that \cite{liuWang2013:agentTypesHLPE} is a nice paper.


\newpage

\addcontentsline{toc}{section}{Bibliography}
\bibliographystyle{alpha}
\bibliography{references.bib}

\end{document}
